\documentclass{article}
%\documentclass{report} % Can be article, report, book, anything you want!

\usepackage{vub} % this will automatically set to A4 paper. We're not in the US.
% if, for some obscure reason, you need another format, please file an issue (see further).

% The `vub` package has many options.
% Please read the documentation at https://gitlab.com/rubdos/texlive-vub if you want to learn about them.

% Feel free to mix and match with other packages.
\setlength{\parskip}{10pt}  % Adjust the space between paragraphs
\usepackage{hyperref}
\usepackage{float}
\usepackage{tabularx}
\usepackage{graphicx}
\graphicspath{{../src/report/images/}{./images/}}

\title{Key-Value Store}
\subtitle{Multicore Programming: Project Erlang}
\faculty{Sciences and Bio-Engineering Sciences} % Note: without the word "Faculty"!
\author{Gérard Lichtert \\
\href{mailto:gerard.lichtert@vub.be}{gerard.lichtert@vub.be}}

\begin{document}
\maketitle

\tableofcontents
\newpage
\raggedright

\section{Overview}
To give a brief summary of the reports' contents, we will first start with the
implementation, which discusses a sharded and cached implementation (henceforth
called the 'sharded' implementation). The sharded implementation distributes
each store in an actor (called bucket) and keeps a central actor to coordinate communication
between clients and the buckets. Another implementation that is discussed is a
worker, sharded and cached implementation (henceforth called the 'worker'
implementation). This implementation is similar to the previous in regards of
the buckets, however whenever a client connects to the server, they get their
own process to communicate with. So this implementation has a one to one process
relation whilst the previous has many to one relation.
\par
These implementations together with the provided central implementation are used
to perform some experiments. The first experiment measures the speedup,
depending on the number of scheduler threads. The second measures throughput
depending on the number of scheduler threads. Both of these experiments use the
worker implementation to analyze the results, however the results for the other
implementations are also available in the dataset.
The last compares the throughput of the different implementations
\section{Implementation}
\subsection{Central implementation}
\subsubsection{Architecture}
To start explaining the implementations, we will start out with the provided
central implementation, which is just a single process containing a dictionary
of dictionary, or a dictionary of buckets, to which the user sends CRUD (Create,
Read, Update, Delete) messages. The single process or 'Server' handles all the
messages and also responds to them after performing the necessary operations. A
diagram showcasing the flow of the messages can be seen in
\autoref{fig:central-implementation}.
\begin{figure}[h!]
	\centering
	\includegraphics[width=0.2\textwidth]{central.png}
	\caption{Message flow of the Central implementation}
	\label{fig:central-implementation}
\end{figure}
As this implementation was meant to be improved upon for scalability, we will not be
answering the scalability questions for this implementation.
\newpage
\subsection{Sharded-Cached implementation}
\subsubsection{Architecture}
To split the central implementation up, instead of having a nested dictionary,
we create a process per bucket, so each bucket process will hold a
dictionary or the Key-Value store and the central process will keep track of the process ID's in its
dictionary to address each bucket. This means that while the central process still has to handle all
communication from the clients, it no longer has to manage the data by itself
only the addresses of the buckets.
The only responsability that the server has is to create the buckets when needed
and forward the messages of the client to the appropiate bucket. The
responsability of the buckets is to perform the operations that the server
forwards to it and to respond to the clients. A diagram showing the flow of
communication can be seen in \autoref{fig:sharded-implementation}.\\
\begin{figure}[h]
	\centering
	\includegraphics[width=0.3\textwidth]{sharded-implementation.png}
	\caption{Message flow of the Sharded implementation}
	\label{fig:sharded-implementation}
\end{figure}
To improve on the performance of the implementation we also add a caching
mechanism that remembers the last bucket addressed in the server. If a message
is to be forwarded to the same bucket twice or more in a row, the server no
longer needs to look the bucket up since it already has the address.
This mechanism is also applied in the buckets for retrieval operations, where it
remembers the last value associated with the last lookup.
\subsubsection{Scalability}
This implementation allows the system to scale with the amount of key-value
stores that exist, since each gets its own process to live in, and the load gets
split to the buckets themselves. The best case would be if the clients operate
on different buckets and the worst case would be if all clients operate on the
same bucket. However since the bottleneck is the server, the load on the buckets
will at most be the same as the server.
\subsection{Worker-Sharded-Cached implementation}
To improve on the Sharded implementation, we need to solve the bottleneck in the
server. Currenty all communication of all the clients go through the server and
get forwarded to the appropiate buckets which is cumbersome for the server. To
aleviate the server we create a worker process for each client that connects to
the server, and in doing so, each client will have its own worker to which it
sends messages to. \\
The worker will in turn forward the messages to the appropiate bucket. The only
issue with this implementation is that when a client creates a bucket, the
bucket must also be available to all the other clients connected to the service.
To solve this we add the responsability of replicating the address of the bucket
to the server. The server keeps track of all existing worker processes and
whenever one of them send a replication message (when a new bucket is created),
it broadcasts it to the rest of the existing worker processes, which add the
bucket to their dictionary of buckets. An example of this is shown in the
diagram in \autoref{fig:worker-bucket-creation-process}.\par
\begin{figure}[H]
	\centering
	\includegraphics[width=0.7\textwidth]{worker-bucket-creation-process.png}
	\caption{Creating a bucket in the Worker implementation}
	\label{fig:worker-bucket-creation-process}
\end{figure}
To keep making use of the caching mechanism, it is moved from the server to the
worker process, since the workers keep track of the buckets that the client
interacts with. Moreover, since it will always be the same client interacting
with the worker, the cache will be hit more frequently. \\
A general flow of messages can be seen in \autoref{fig:worker-implementation}.
\begin{figure}[H]
	\centering
	\includegraphics[width=0.4\textwidth]{worker-implementation.png}
	\caption{Message flow of the Worker implementation}
	\label{fig:worker-implementation}
\end{figure}
\subsection{Scalability}
This implementation is a lot more scalable than the previous implementation. It
now allows the system to also scale with the amount of clients and thus allows
for messages to be processed in parallel. The best case would be if the client
on different buckets, allowing the load to be distributed across the buckets.
The worst case would be if all the clients operate on a single bucket, creating
a bottleneck in the bucket. Since this implementation allows for the
parallelisation of the sent messages, the bottleneck there is now the speed of
the processing of the messages. Another bottleneck would be the buckets
themselves, if they receive more messages than they can handle. For this we
could explore further sharding of the bucket, or replication as well as pla:!cing
the bucket behind a load balancer.
\newpage
\section{Evaluation}
\subsection{Experimental set-up}
For the experiments I used two devices. Firstly the Firefly machine of the VUB
and my own laptop henceforth called "Omen" to set up the experiments for Firefly
as well as make the necessary scripts and programs to run the experiments on
firfly and analyze the data automatically. The specifications of both devices
are found in \autoref{table:hardware}
\begin{table}[H]
	\centering
	\begin{tabularx}{\textwidth}{c | X | X}
		Device                       & Firefly                                                   & Omen HP laptop            \\
		\hline
		\textbf{Hardware}            &                                                           &                           \\
		CPU                          & AMD Ryzen Threadripper 3990X Processor
		(64 cores / 128 threads @
		2.9 GHz Base, 4.3 GHz boost) & Intel Core i7-9750H (6 cores / 12 threads) @ 2.6 GHz, 4.5
		Ghz boost                                                                                                            \\
		\hline
		RAM                          & 128 GB (DDR4 3200 MHz)
		                             & 16GB (DDR4 2667 MHz)                                                                  \\
		\hline
		\textbf{Software}            &                                                           &                           \\
		OS                           & Ubuntu 24.04.1                                            & Microsoft Windows 11 Home \\
		\hline
		Erlang/OTP version           & 25.3.2.8                                                  & 27                        \\
	\end{tabularx}
	\caption{Hardware used for the experiments}
	\label{table:hardware}
\end{table}
For the experiments I used a Powershell script and a Bash script to start an
erlang process to perform an experiment and benchmark it 30 times. This was done
for each implementation, so for the central, sharded and worker implementation
and for three different scenarios. The scenarios were: A read only scenario,
where clients only send retrieve messages to the server. A read write scenario
where clients performed 50-50 writes and reads. Finally also a mixed scenario
where the balance was 90\% reads and 10\% writes. Note that only the time
elapsed for performing these operations was measured and thus the buckets were
already existing prior to starting the benchmark. \par
Each experiment has 100 clients performing operations on the system. Each client
will do 1000 operations per scenario and 2000 operations in the read-write
scenario. There will always be 1000 buckets, each holding 100 keys
\subsection{Experimental methodology}
To measure the time elapsed we used the wall clock time on the client side and we performed the
experiment 64 times per scenario and per implementation on the Omen device, once
per amount of scheduler threads (so 1-64). On Firefly we performed the
experiment 128 times per scenario, per implementation since it can use a lot
more threads.
\subsection{Experiments}
For the first experiment we will be comparing the throughput of the different
implementations on the maximum amount of scheduler threads (12 for Omen and 128
for Firefly).
\section{Insight}

\end{document}
